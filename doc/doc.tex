\documentclass[12pt, letterpaper]{article}
\usepackage[utf8]{inputenc}
\usepackage{cite}
\usepackage{float}
\usepackage{tikz}
\usepackage{hyperref}
\usepackage[newfloat]{minted}
\usepackage{caption}
\usepackage{dirtree}
\tolerance=1
\emergencystretch=\maxdimen
\hyphenpenalty=10000
\hbadness=10000

\newenvironment{code}{\captionsetup{type=listing}}{}
\SetupFloatingEnvironment{listing}{name=Source Code}

\graphicspath{{img/}}

\title{Simulation of TLS (42)}
\author{Leskovar Lukas Andreios (KatNr), 5BHIF}
\date{March 2022}

\begin{document}

\begin{titlepage}
\maketitle
\end{titlepage}

\tableofcontents
\newpage

\section{Introduction}
The goal of this project was to simulate communication over Transport Layer Security (TLS) by implementing the Diffie-Hellman Internet Key Exchange (IKE). Any further communication was to be encrypted by a symmetric encryption algorithm.

\section{Implementation}


\subsection{TLS 1.0}


\subsubsection{Key Generation}


\subsubsection{Handshake}


\section{Software Architecture}

\subsection{Technologies}

\begin{table}[h]
	\centering
	\begin{tabular}{l|l}
		Purpose                       & Technology \\ \hline
		Build Tool				 	  & Meson	   \\
		Command line interface	      & CLI11      \\
		Configuration files           & json       \\
		Data serialization            & Protobuf   \\
		Logging                       & spdlog     \\
		Network Communication         & asio       \\
		Programming Languages		  & C++ 17 \\
		Encryption 					& plusaes \\
		Hashing 					& PicoSHA2 \\
		Large Integer Values & BigInt \\
	\end{tabular}
	\caption{This table lists all the technologies used in this project.}
\end{table}

\subsection{Classes}

\subsection{Communication}


\section{Description of code-blocks}


\subsection{Asio}
Network communication between client and server is established by utilizing asio.

\subsubsection{Client Connection}

\begin{code}
	\begin{minted}{cpp}
asio::io_context io_context;
asio::ip::tcp::resolver resolver(io_context);
asio::ip::tcp::socket socket(io_context);
asio::ip::tcp::resolver::results_type endpoints = 
  resolver.resolve(host, port);

asio::connect(socket, endpoints);

	\end{minted}
	\caption{Creation of socket connection on client side.}
	\label{clientConnection}
\end{code}

\pagebreak

\subsubsection{Server Connection}

\begin{code}
	\begin{minted}{cpp}
asio::io_context io_context{1};
asio::ip::tcp::acceptor acceptor{
  io_context, asio::ip::tcp::endpoint(asio::ip::tcp::v4(), port)
};
acceptor.async_accept(
[this](const std::error_code& ec, asio::ip::tcp::socket socket) {
  if (!ec) {
    // handle socket
  } else {
    // throw error
  }
});
	\end{minted}
	\caption{Server asynchronously waiting for client connections.}
	\label{serverConnection}
\end{code}

\subsection{Protobuf}
Any data to be sent over TCP is serialized using Google Protobuf.

\subsubsection{Message Serialization}
\begin{code}
	\begin{minted}{cpp}
void Pipe::send(google::protobuf::Message& message) {
  u_int64_t message_size{message.ByteSizeLong()};
  asio::write(*socket, asio::buffer(&message_size, sizeof(message_size)));
	
  asio::streambuf buffer;
  std::ostream os(&buffer);
  message.SerializeToOstream(&os);
  asio::write(*socket, buffer);
}
	\end{minted}
	\caption{Serialization of protobuf messages.}
	\label{pipeSend}
\end{code}

\subsubsection{Message De-serialization}
\begin{code}
	\begin{minted}{cpp}
void Pipe::receive(google::protobuf::Message& message) {
  u_int64_t message_size;
  asio::read(*socket, asio::buffer(&message_size, sizeof(message_size)));
	
  asio::streambuf buffer;
  asio::streambuf::mutable_buffers_type bufs = buffer.prepare(message_size);
  buffer.commit(asio::read(*socket, bufs));
	
  std::istream is(&buffer);
  message.ParseFromIstream(&is);
}
	\end{minted}
	\caption{De-serialization of protobuf messages.}
	\label{pipeReceive}
\end{code}


%Pipe Klasse 

\subsection{TLS Handshake}
Whenever a new message is received the TLS\_Handshake\_Agent class is responsible for handling and responding to any handshake related message.

\subsection{Message Handling}

\begin{code}
	\begin{minted}{cpp}
void TLS_Handshake_Agent::handle_message(tls::MessageWrapper message) {    
  tls::MessageType messageType = message.type();
	
  if (messageType == tls::MessageType::CLIENT_HELLO) {
    receive_client_hello();
    
  } else if (messageType == tls::MessageType::SERVER_HELLO) {
    receive_server_hello(message);
		
  } else if (messageType == tls::MessageType::CERTIFICATE) {
    receive_certificate(message);
		
  } else if (messageType == tls::MessageType::SERVER_HELLO_DONE) {
    receive_server_hello_done();
		
  } else if (messageType == tls::MessageType::CLIENT_KEY_EXCHANGE) {
    receive_client_key_exchange(message);
	
  } else if (messageType == tls::MessageType::CHANGE_CIPHER_SPEC) {
    partnerEncrypted = true;
		
  } else if (messageType == tls::MessageType::FINISHED) {
    if (!partnerEncrypted) {
      session->send(Messagebuilder::build_abort_message());
      throw std::runtime_error("Partner did not send ChangeCipherSpec");
    }
    receive_finished(message);
		
  } else if (messageType == tls::MessageType::ABORT) {
    currentState = State::UNSECURED;
    throw new std::runtime_error("TLS connection aborted");
  } else {
    spdlog::error("Unknown message type: {}", messageType);
  }
}
	\end{minted}
	\caption{TLS Handshake Agent handling a message.}
	\label{messageHandling}
\end{code}


\subsection{External Libraries}
\label{extBib}

\subsubsection{CLI11}
CLI11 implements a basic Command Line Interface (CLI) where users are able to specify parameters relevant for the program. 


\subsubsection{spdlog}
To log important information the logging library spdlog is employed.

\begin{code}
	\begin{minted}{cpp}
spdlog::trace("Trace bugs during development");
spdlog::debug("Debug messages");
spdlog::info("User-facing messages");
spdlog::warn("Potential errors");
spdlog::error("Errors");
spdlog::critical("Critical errors");
	\end{minted}
	\caption{Usage of different log types.}
	\label{spdlog}
\end{code}



\subsubsection{JSON}
Any further information, e.g. prime number for Diffie-Hellman IKE, is stored in a .json file which is read as follows. This is done using nlohman/json

\begin{code}
	\begin{minted}{cpp}
void TLS_Handshake_Agent::read_primes_json(
  std::string filename, 
  int id, 
  BigInt& g, 
  BigInt& p
) {
  std::ifstream file(filename);
  if (!file.is_open()) {
    throw new std::runtime_error("Error opening file");
  }
  nlohmann::json primes;
  file >> primes;
  file.close();
  g = int(primes["groups"][id]["g"]); 
  p = std::string(primes["groups"][id]["p_dec"]);
}
	\end{minted}
	\caption{nlohman reading the generator $g$ and prime number $p$.}
	\label{json}
\end{code}


\subsubsection{plusaes}
The header-only library plusaes is used to encrypt/decrypt messages.

\begin{code}
	\begin{minted}{cpp}
std::vector<unsigned char> key(32);
	
size = plusaes::get_padded_encrypted_size(message.size());
std::vector<unsigned char> encrypted(size);
	
plusaes::encrypt_cbc(
  (unsigned char*)message.data(), message.size(), 
  &key[0], key.size(), 
  &iv, 
  &encrypted[0], encrypted.size(), 
  true
);

	\end{minted}
	\caption{Plusaes encrypting a message}
	\label{plusaes}
\end{code}

\subsubsection{picoSHA2}
Hashing is done using the header-only library picoSHA2.

\begin{code}
	\begin{minted}{cpp}
std::vector<unsigned char> key(32);
picosha2::hash256(key_string.begin(), key_string.end(), key);
	\end{minted}
	\caption{Plusaes encrypting a message}
	\label{picosha2}
\end{code}

\section{Usage}
\label{usage}

\subsection{Command Line Arguments}

\paragraph{-n, --hostname}
Hostname of the server (default: localhost)

\paragraph{-i, --ip}
IPv4 address of the server (to be preferred over hostname)

\paragraph{-p, --port}
The port of the server (default: 4433)

\paragraph{-l, --log-level}
Log-Level of the application (default: info)


\newpage

\section{Project Structure}
\dirtree{%
	.1 /.
	.2 LICENSE.
	.2 meson\_options.txt.
	.2 meson.build.
	.2 README.md.
	.2 CHANGELOG.org.
	.2 modp\_primes.json.
	.2 include.
	.3 tls\_client.h.
	.3 tls\_server.h.
	.2 src.
	.3 client.cpp.
	.3 server.cpp.
	.3 tls\_client.cpp.
	.3 tls\_server.cpp.
	.2 doc.
	.3 doc.tex.
	.3 references.bib.
	.3 doc.pdf.
	.2 tls\_util.
	.3 include.
	.4 BigInt.hpp.
	.4 picosha2.h.
	.4 plusaes.hpp.
	.4 messagebuilder.h.
	.4 pipe.h.
	.4 session.h.
	.4 tls\_handshake\_agent.h.
	.4 tls\_observer.h.
	.3 src.
	.4 Message.proto.
	.4 pipe.cpp.
	.4 session.cpp.
	.4 tls\_handshake\_agent.cpp.
	.3 meson.build.
	.2 build.
}\hfill

% .bib include & references
\newpage
\bibliography{references}
\bibliographystyle{plain}
\end{document}